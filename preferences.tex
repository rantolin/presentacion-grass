%% PAQUETES
\usepackage[utf8]{inputenx}
\usepackage[spanish]{babel}
\usepackage{amsmath}
\usepackage{amsthm}
\usepackage{amssymb}
\usepackage{amsfonts}   
\usepackage{pifont}     % Para tickYes y tickNo
\usepackage{graphicx}
\usepackage{epsfig}
\usepackage{pdflscape}
\usepackage{enumerate}
\usepackage{color}
\usepackage{beamerfoils}
\usepackage{beamerprosper}
\usepackage{xspace}
\usepackage{booktabs}
\usepackage{multicol}
\usepackage{multirow}
\usepackage{tikz}
\usepackage{pgflibraryshapes}
\usepackage{wasysym}
\usepackage{float}
\usepackage{listings}

%% CONFIGURACIÓN DE TIKZ
% Librerías cargadas
\usetikzlibrary{arrows}
%\usetikzlibrary{calc,fadings,decorations.pathreplacing,arrows,intersections,
%    decorations.markings,external,shapes}

% Configuración de Tikz y Pgfplot
\tikzset{%
    >=stealth, % option for nice arrows
    inner sep=0pt,%
    outer sep=3pt,%
    mark coordinate/.style={inner sep=0pt,outer
    sep=0pt,minimum size=3pt,
    fill=black,circle}%
    }

% Librerías de tiKz
    %\usetikzlibrary{}

    % Define block styles
\tikzstyle{io} = [trapezium,trapezium left angle=70,trapezium right angle=-70,
  minimum height=0.9cm, draw, fill=blue!20, text width=4.5em, 
  text badly centered, node distance=2cm, inner sep=0pt]
\tikzstyle{decision} = [diamond, draw, fill=blue!20, text width=5.5em, 
  text badly centered, node distance=2cm, inner sep=0pt]
\tikzstyle{block} = [rectangle, draw, fill=blue!20, text width=5em, 
  text centered, rounded corners, minimum height=4em]
\tikzstyle{line} = [draw, -latex']
\tikzstyle{cloud} = [draw, ellipse,fill=red!20, node distance=2cm, 
    minimum height=2em]


%% TEMAS BASE
\usefonttheme{structurebold}
\usecolortheme{orchid}
\usetheme[hideothersubsections,left,width=.13\textwidth]{Goettingen}

% \usetheme{Boadilla}
% \setbeamertemplate{headline}{
%   \begin{beamercolorbox}{section in head/foot}
%   \vskip2pt\insertnavigation{\paperwidth}\vskip2pt
%   \end{beamercolorbox}
%   \includegraphics[width=\paperwidth]{images/ucm_logo}
% }
% \usecolortheme{lily}

\useinnertheme{circles}

\definecolor{ZurichBlue}{rgb}{.255,.41,.884}
\setbeamercolor{alerted text}{fg=ZurichBlue}
\definecolor{Comentario}{rgb}{0.0,0.5,0.0}

%% CONFIGURACIÓN GEENERAL DE HIPERVÍNCULOS
\hypersetup{%
    baseurl={roberto.antolin@forestry.gsi.gov.uk},
    pdftitle={Filtro de datos LiDAR con GRASS},%
    pdfauthor={Roberto Antol\'{i}n S\'anchez},%
    pdfcreator={Roberto Antol\'{i}n S\'anchez},%
    pdfproducer=PDFLaTeX,%
    pdfsubject={LiDAR GRASS data filtering},%
    pdfkeywords={filtros, splines, LiDAR, GRASS}%
  }

\mode<presentation>{\hypersetup{pdfpagemode=FullScreen}}

%% COMANDOS
% Tipos de datos
\newcommand{\file}[1]{\path{#1}\xspace} %Referencias a ficheros
\newcommand{\extr}[1]{\emph{#1}\xspace} %Palabras extranjeras
\mode<article>{\newcommand{\soft}[1]{\textsf{#1}\xspace} }%Referencias a software
\mode<presentation>{\newcommand{\soft}[1]{\textrm{#1}\xspace} }%Referencias a software
\newcommand{\sigl}[1]{\textsc{#1}\xspace} %Siglas
\newcommand{\nota}[1]{\textcolor{nota}{�#1?}\xspace}
\newcommand{\frametitlee}[1]{\mode<presentation| handout| article:0>{\frametitle{#1}}}
\newcommand{\figuree}[2]{
  \mode<article>{
    \begin{figure}[H]
    \centering
    }
    #1
    \mode<article>{
    \caption{#2}
    \end{figure}
  }
}

% Atajos
\newcommand{\inet}{\extr{Internet}}
\newcommand{\html}{\sigl{html}}
\newcommand{\web}{\extr{web}}
\newcommand{\software}{\extr{software}}
\newcommand{\hardware}{\extr{hardware}}
\newcommand{\tickNo}{\color{red}\ding{55}}
\newcommand{\tickYes}{\color{green!50!black}\checkmark}

%% PREFERENCIAS PARA LISTINGS (código)
\lstdefinestyle{mio}{
  basicstyle=\ttfamily\scriptsize,%
  keywordstyle=\bfseries\color{ZurichBlue},%
  commentstyle=shape\color{green},%
  stringstyle=\color{red!50!black},
}

\lstdefinestyle{latex}
{
language={[latex]tex},
keywordstyle=\color{resalta!50!black},
commentstyle=\color{fondoO!50!green}\itshape
}

\lstdefinestyle{shell}{
  language=sh,
  basicstyle=\ttfamily\scriptsize,%
  morekeywords={lasinfo,lasdiff,lasmerge,las2txt,txt2las,las2ogr,laszip,
    lasview, lasboundary, lasgrid, lasduplicate, lasoverlap,
    las2dem, las2las, lastile, lasground, lasheight, lasclassify},
  keywordstyle=\bfseries\color{ZurichBlue},
  commentstyle=\color{green!35!black}%
}

\lstdefinestyle{grass}{                                                                     
  language=sh,
  basicstyle=\ttfamily\scriptsize,%
  morekeywords={v.lidar.edgedetection,v.lidar.growing, v.lidar.correction,v.surf.bspline,v.outlier},
  keywordstyle=\bfseries\color{ZurichBlue},classoffset=1,
  commentstyle=shape\color{green!75!black},%
  morekeywords={GRASS6 (stuttgart):, >}
  keywordstyle=\color{green!50!black}, classoffset=0
}

%
\lstdefinestyle{matlab}{
  basicstyle=\scriptsize\ttfamily,
  breaklines=true,
  commentstyle=\color{Comentario},
  keywordstyle=\bfseries\color{blue},
  language=Matlab,
  %xleftmargin=\parindent,
  identifierstyle=\bfseries
}

\lstset{style=mio,
   tabsize=4,%
   inputencoding=utf8,%latin1,%
   extendedchars=true,%
   breaklines=true,%
   showstringspaces=false,
   backgroundcolor=\color{gray!20!white}
   }

%% DATOS DEL DOCUMENTO
\title[Tutorial: procesar datos LiDAR con GRASS]
    {Tutorial: procesar datos LiDAR con GRASS}

\author[R. Antolín]
    {Roberto Antolín Sánchez}

\institute[FR]{
  Investigador y desarrollador \\
	Forest Research \\
  Northern Research Station \\
	Roslin EH25 9SY \\
	\texttt{roberto.antolin@forestry.gsi.gov.uk}\\
}

%% FECHA DE LA PRESENTACIÓN
\date[15-03-2014]{15 de Marzo de 2014}

\AtBeginSection[]
{
  \begin{frame}[shrink=20]
    \frametitle{Table of Contents}
    \tableofcontents[currentsection,hidesubsections]
  \end{frame}
}

\begin{document}
  %%==================================================================F PORTADA
\begin{frame}[label=portada]
  \titlepage
\end{frame}%Fin del frame
%%==================================================================FRONT PAGE AND TOC
%% For article only
%\mode<presentation:0>{\thispagestyle{empty}\maketitle}
%
%% For presentation only
%\mode<presentation| article:0| handout:0>{
%    \begin{frame}<article:0>[label=portada]
%    \titlepage
%    \end{frame}%Fin del frame
%}
%
%% For handout only
%\mode<handout>{
%  \begin{frame}[label=portada]
%    \maketitle
%  \end{frame}
%}
%
%%% TABLE OF CONTENTS
\begin{frame}[label=toc]
   \mode<article:0>{\frametitle{Contents}}
   \mode<presentation>{\small}
   \tableofcontents[hidesubsections]
\end{frame}
%
%%==================================================================S LOS DATOS
\section{Los datos}
%%==================================================================F 
\begin{frame}
\frametitle{La nube de puntos}
Metadatos
\begin{itemize}
 \item Sensor LiDAR: \alert{Optech ATLM}
 \item Datos disponibles: Primer y último impulso.
 \item Localización: Centro de Stuttgart (estación f.c.)
 \item Resolución: $0.91$ puntos$/m^2$ 
 \item Resolución media: $0.87$ puntos$/m^2$
 \item Numero de puntos: 
  \begin{enumerate}
    \item $259030$ de primeros retornos
    \item $259030$ de últimos retornos
   \end{enumerate}
\end{itemize}
\end{frame}
%%==================================================================F 
\pgfdeclareimage[width=0.5\textwidth]{stuttgart}{images/stuttgart_google}
\begin{frame}
\frametitle{Estación f.c. de Stuttgart}
 \begin{columns}
  \begin{column}{0.45\textwidth}
	Coordenadas UTM-32 (MBR):
	\begin{itemize}
	 \item Norte = $5403764~m$
	 \item Sur = $5403188~m$
	 \item Oeste = $513632~m$
	 \item Este = $513118~m$
	\end{itemize}
   \end{column}
  \begin{column}{0.55\textwidth}
    \begin{center}
	\begin{tikzpicture}
    	  \pgftext[bottom,left,at={\pgfpointxy{0}{0}}]{\pgfuseimage{stuttgart}}
	\end{tikzpicture}
	\tiny{\textcopyright 2008 Google -- imágenes}
    \end{center}
   \end{column}
 \end{columns}
\end{frame}
%%==================================================================F 
\pgfdeclareimage[width=0.7\textwidth]{las}{images/las_format}
\begin{frame}
\frametitle{Estandard binario \emph{LAS} del ASPRS:}
\begin{itemize}
 \item Encabezado público
 \item Registros de longitud variable
 \item Puntos: X, Y, Z, Intensidad, clasificación, dirección de la pasada del vuelo,...
    \begin{center}
	\begin{tikzpicture}
    	  \pgftext[bottom,left,at={\pgfpointxy{0}{0}}]{\pgfuseimage{las}}
	\end{tikzpicture}
   \end{center}
 \item Para un conversor \alert{abierto} del estándar \emph{LAS}:
\beamergotobutton{\url{http://liblas.org/}}
\end{itemize}
\end{frame}
%%==================================================================F 
\frame[containsverbatim]{%
\frametitle {Posible formato ASCII}
\begin{center}
\begin{verbatim}
              X          Y       Z
          513628.82|5403190.04|291.40
          513628.85|5403191.47|291.24
          513628.95|5403192.93|294.31
          513628.99|5403194.45|294.17
          513628.97|5403196.05|291.26
          513629.01|5403197.58|291.24
          513629.05|5403199.10|291.25
          513629.09|5403200.53|291.26
                        ...
\end{verbatim}
\end{center}
}
%%==================================================================S
\section{Preprocesado de datos}
%%==================================================================F
\begin{frame}
 \frametitle{Geographic Resources Analysis Support System\hfill 
   \includegraphics[width=0.7cm]{images/grasslogo_transp_big.png}} 
 \begin{itemize}
    \item Conocido como \alert<1>{GRASS}: GIS para manipulación y 
      análisis de datos geoespaciales.
    \item Como es un programa de \alert{códico abierto} con un 
      montón de librerias, es muy adecuado para la implementación 
      de nuestras propias herramientas.
    \item Escrito en C/C++, lo que permite implementar programas 
      computacionalmente pesados $\Rightarrow$ \alert{ideal} 
      para trabajar con LiDAR.
    \item GRASS $\geq$ 6 incluye indexación topológica $\Rightarrow$ 
      \alert{fatal} para trabajar con LiDAR.
    \item Hay versiones para GNU/Linux, MS-Windows, Mac-OSX, SUN, etc. 
      \alert{\textexclamdown\textexclamdown Todos lo pueden utilizar!!}
    \item Más información en: \beamergotobutton{\url{http://grass.osgeo.ogr}} 
  \end{itemize}
\end{frame}
%%==================================================================Sb
\subsection{Importando nube de puntos}
%%==================================================================F 
\defverbatim[colored]\vlidar{
\begin{lstlisting}[style=shell]
# Mostrar la informacion del LAS
v.in.lidar -p input=CSite4_orig.las
# Crear una nueva ``localizacion'' con la informacion del LAS
v.in.lidar -i input=CSite4_orig.las location=Stuttgart
# Cerrar y abrir GRASS en la nueva localizacion
# Importar el archivo LAS sin topologia ni tabla asociada
v.in.lidar -tb input=CSite4_orig.las output=stuttgartLAS
\end{lstlisting}
}
\begin{frame}
 \frametitle{Importar como .LAS: \LARGE\path{v.in.las}}
 \vlidar
\end{frame}
%%==================================================================F 
\begin{frame}
  \frametitle{Importar como LAS: \LARGE\path{v.in.las}}
 \begin{center}
 \includegraphics[height=50mm]{images/vinascii_strips}
 \end{center}
\end{frame}
%%==================================================================F 
\defverbatim[colored]\head{
\begin{lstlisting}[style=shell]
  $ # Copiar python en ``/Documents and Settings/aula1/Desktop/taller LiDAR/''
  $ # Descargar el conversor de datos de http://tinyurl.com/ca3vdfb
  $ # Formato ASCII de los datos
  $ python isprs2lf.py CSite4_orig.txt
  $ head CSite4_orig.txt
\end{lstlisting}
}
\begin{frame}[fragile,shrink=2]
\frametitle {Posible formato ASCII}
\head
\begin{center}
\begin{verbatim}
              X          Y       Z
          513628.82|5403190.04|291.40
          513628.85|5403191.47|291.24
          513628.95|5403192.93|294.31
          513628.99|5403194.45|294.17
          513628.97|5403196.05|291.26
\end{verbatim}
%          513629.01|5403197.58|291.24
%          513629.05|5403199.10|291.25
%          513629.09|5403200.53|291.26
%                        ...
\end{center}
\end{frame}
%%==================================================================F 
\defverbatim[colored]\extension{
\begin{lstlisting}[style=shell]
GRASS6: > # Primero conocemos la extension del archivo
GRASS6: > r.in.xyz -sg in=CSite4_orig_all.txt
GRASS6: > # Fijo la region con la que trabajar
GRASS6: > g.region n=5403764 s=5403188 e=513632 w=513118
GRASS6: > # Genero una visualizacion de las pasadas
GRASS6: > # considerando para cada celda el numero de puntos
GRASS6: > r.in.xyz in=CSite4_orig.txt out=pasadas method=n fs='|'
\end{lstlisting}
}
\begin{frame}
 \frametitle{Importar como raster: \LARGE\path{r.in.xyz}}
 \extension
\end{frame}
%%==================================================================F 
\defverbatim[colored]\xyz{
\begin{lstlisting}[style=shell]
# Generar un raster con la informacion LiDAR
r.in.xyz in=CSite4_orig_first.txt out=stuttgart_first fs='|' x=1 y=2 z=3
r.in.xyz in=CSite4_orig_last.txt out=stuttgart_last fs='|' x=1 y=2 z=3
\end{lstlisting}
}
\begin{frame}
 \frametitle{\LARGE\path{r.in.xyz}}
 \xyz
 \pause
 \begin{center}
 \includegraphics[height=50mm]{images/rinxyz_first}~
 \includegraphics[height=50mm]{images/rinxyz_last}
 \end{center}
\end{frame}
%%==================================================================F 
\defverbatim[colored]\vascii{
\begin{lstlisting}[style=shell]
# Generar dos vectoriales con la informacion LiDAR
# de ultimo y primer impulso
v.in.ascii -tbzr in=CSite4_orig_first.txt out=stuttgart_first fs='|' x=1 y=2 z=3
v.in.ascii -tbzr in=CSite4_orig_last.txt out=stuttgart_last fs='|' x=1 y=2 z=3
\end{lstlisting}
}
\begin{frame}
 \frametitle{Importar como vectorial: \LARGE\path{v.in.ascii}}
 \vascii
 \pause
 \begin{center}
 \includegraphics[height=40mm]{images/vinascii_strips}
 \end{center}
\end{frame}
%%==================================================================Sb
\subsection{Eliminando puntos groseros}
%%==================================================================F 
\defverbatim[colored]\voutlier{
\begin{lstlisting}[style=shell]
# Limpiamos el primer impulso
v.outlier in=stuttgart_first out=station_first \
> outlier=station_first_out soe=10 son=10 thres_o=25
# Limpiamos el ultimo impulso
v.outlier in=stuttgart_last out=station_last \
> outlier=station_last_out soe=10 son=10 thres_o=30
\end{lstlisting}
}
\begin{frame}[shrink=05]
 \frametitle{\path{v.outlier}}
\begin{beamerboxesrounded}[shadow=true]{\textbf{\path{v.lidar.edgedetection}}\texttt{ input=name output=name outlier=name [soe=value]
[son=value] [lambda\_i=value] [thres\_o=value]}}
\begin{itemize}
 \item \textbf{input}: name of the input vector map
 \item \textbf{output}: name of the output vector map
 \item \textbf{soe}: Interpolation spline step value in e-w direction
 \item \textbf{son}: Interpolation spline step value in n-s direction
 \item \textbf{lambda\_i}: Thychonov regularization weigth. (default: 0.1)
 \item \textbf{thres\_o}: Threshold for the outliers. (default: 50)
\end{itemize}
\end{beamerboxesrounded}
\voutlier
\end{frame}
%%==================================================================F 
\begin{frame}
 \frametitle{Visualización de \LARGE\path{v.outlier}}
 \begin{center}
 \includegraphics[height=50mm]{images/voutlier_first}~
 \includegraphics[height=50mm]{images/routlier_first}
 \end{center}
\end{frame}
%%==================================================================F 
\begin{frame}
 \frametitle{Visualización de \LARGE\path{v.outlier}}
 \begin{center}
 \includegraphics[height=50mm]{images/voutlier_last}~
 \includegraphics[height=50mm]{images/routlier_last}
 \end{center}
\end{frame}
%%==================================================================S
\section{Clasificación de la nube de puntos}
%%==================================================================Sb
\subsection{Detección de borde de objetos}
%%==================================================================F 
\defverbatim[colored]\vedge{
\begin{lstlisting}[style=shell]
# Deteccion de bordes
v.lidar.edgedetection in=station_last see=4 sen=4 lambda_g=0.1 \
> out=station_edge
\end{lstlisting}
}
\begin{frame}[shrink=02]
 \frametitle{\path{v.lidar.edgedetection}}
\begin{beamerboxesrounded}[shadow=true]{\textbf{\path{v.lidar.edgedetection}}\texttt{ input=name output=name first=name see=value sen=value [lambda\_g=value] [tgh=value] [tgl=value] [theta\_g=value] [lambda\_r=value]}}
\begin{itemize}
 \item \textbf{input}: name of the input vector map
 \item \textbf{output}: name of the output vector map
 \item \textbf{see}: Interpolation spline step value in e-w direction
 \item \textbf{sen}: Interpolation spline step value in n-s direction
 \item \textbf{lambda\_g}: Regularization weight in gradient evaluation (default: 0.01)
 \item \textbf{tgh}: High gradient threshold for edge classification (default: 6)
 \item \textbf{tgl}: Low gradient threshold for edge classification (default: 3)
 \item \textbf{theta\_g}: Angle range for same direction detection (default: 0.26)
 \item \textbf{lambda\_r}: Regularization weight in residual evaluation (default: 2)
\end{itemize}
\end{beamerboxesrounded}
\vedge
\end{frame}
%%==================================================================F 
\pgfdeclareimage[width=0.5\textwidth]{edge}{images/edge}
\defverbatim[colored]\dedge{
\begin{lstlisting}[style=shell]
$ d.vect station_edge size=1 cat=1 color=yellow
$ d.vect station_edge size=1 cat=2 color=red
\end{lstlisting}
}
\begin{frame}
 \frametitle{Visualización de \LARGE\path{v.lidar.edgedetection}}
    \dedge
\begin{columns}
  \begin{column}{0.65\textwidth}
    \begin{center}
	\begin{tikzpicture}
    	  \pgftext[bottom,left,at={\pgfpointxy{0}{0}}]{\pgfuseimage{edge}}
	\end{tikzpicture}
   \end{center}
  \end{column}
  \begin{column}{0.35\textwidth}
    \begin{enumerate}
      \item \textcolor{red}{Borde}
      \item \textcolor{yellow!95!black}{Terreno}
    \end{enumerate}
  \end{column}
\end{columns}
\end{frame}
%%==================================================================Sb
\subsection{Relleno de los límites de los objetos}
%%==================================================================F 
\defverbatim[colored]\vgrow{
\begin{lstlisting}[style=shell]
# IMPORTANTE: Disminuir primero la resolucion
g.region -p res=2
v.lidar.growing in=station_edge out=station_grow \
> first=station_first
\end{lstlisting}
}
\begin{frame}[shrink=03]
 \frametitle{\path{v.lidar.growing}}
\begin{beamerboxesrounded}[shadow=true]{\textbf{\path{v.lidar.growing}}\texttt{ input=name output=name first=name [tj=value] [td=value]}}
\begin{itemize}
 \item \textbf{input}: name of the input vector map
 \item \textbf{output}: name of the output vector map
 \item \textbf{first}: name of the input vector map of the first pulse points
 \item \textbf{tj}: Threshold in planimetric units for considering two measurements as corresponding first and double pulse. (default
0.2)
 \item td: Threshold in height units for classifying a point as double pulse. (default: 0.6)
\end{itemize}
\end{beamerboxesrounded}
\vgrow
\end{frame}
%%==================================================================F 
\pgfdeclareimage[width=0.45\textwidth]{grow}{images/grow}
\defverbatim[colored]\dgrow{
\begin{lstlisting}[style=shell]
$ d.vect station_grow layer=2 size=1 cat=1 color=yellow
$ d.vect station_grow layer=2 size=1 cat=2 color=green
$ d.vect station_grow layer=2 size=1 cat=3 color=blue
$ d.vect station_grow layer=2 size=1 cat=4 color=red
\end{lstlisting}
}
\begin{frame}
 \frametitle{Visualización de \LARGE\path{v.lidar.growing}}
 \dgrow
\begin{columns}
  \begin{column}{0.65\textwidth}
    \begin{center}
	\begin{tikzpicture}
    	  \pgftext[bottom,left,at={\pgfpointxy{0}{0}}]{\pgfuseimage{grow}}
	\end{tikzpicture}
   \end{center}
  \end{column}
  \begin{column}{0.35\textwidth}
	\begin{enumerate}
	 \item \textcolor{yellow!95!black}{Terreno}
	 \item \textcolor{green}{Terreno doble eco}
	 \item \textcolor{blue!90!black}{Objeto doble eco}
	 \item \textcolor{red}{Objeto}
	\end{enumerate}
  \end{column}
\end{columns}
\end{frame}
%%==================================================================Sb
\subsection{Corrección final}
%%==================================================================F 
\defverbatim[colored]\vcorr{
\begin{lstlisting}[style=shell]
v.lidar.correction in=station_grow sce=60 scn=60 lambda_c=2 \
> tcl=0.1 out=station_corr terrain=station_terr
\end{lstlisting}
}
\begin{frame}[shrink=04]
 \frametitle{\path{v.lidar.correction}}
\begin{beamerboxesrounded}[shadow=true]{\textbf{\path{v.lidar.correction}}\texttt{ input=name  output=name  terrain=name  sce=value  scn=value  [lambda\_c=value]  [tch=value]  [tcl=value]}}
\begin{itemize}
 \item \textbf{input}: name of the input vector map
 \item \textbf{output}: name of the output vector map
 \item \textbf{terrain}: name of the terrain output vector map
 \item \textbf{sce}: Interpolation spline step value in e-w direction
 \item \textbf{scn}: Interpolation spline step value in n-s direction
 \item \textbf{lambda\_c}: Regularization weight in reclassification evaluation. (default 1)
 \item \textbf{tch}: High difference threshold (terrain/object). (default 2)
 \item \textbf{tcl}: Low differnce threshold (object/terrain). (default 1)
\end{itemize}
\end{beamerboxesrounded}
\vcorr
\end{frame}
%%==================================================================F 
\pgfdeclareimage[width=0.45\textwidth]{corr}{images/correction}
\defverbatim[colored]\dcorr{
\begin{lstlisting}[style=shell]
$ d.vect station_corr layer=2 size=1 cat=1 color=yellow
$ d.vect station_corr layer=2 size=1 cat=2 color=green
$ d.vect station_corr layer=2 size=1 cat=3 color=blue
$ d.vect station_corr layer=2 size=1 cat=4 color=red
\end{lstlisting}
}
\begin{frame}
 \frametitle{Visualización de \LARGE\path{v.lidar.correction}}
 \dcorr
\begin{columns}
  \begin{column}{0.65\textwidth}
    \begin{center}
	\begin{tikzpicture}
    	  \pgftext[bottom,left,at={\pgfpointxy{0}{0}}]{\pgfuseimage{corr}}
	\end{tikzpicture}
   \end{center}
  \end{column}
  \begin{column}{0.35\textwidth}
	\begin{enumerate}
	 \item \textcolor{yellow!95!black}{Terreno}
	 \item \textcolor{green}{Terreno doble eco}
	 \item \textcolor{blue!90!black}{Objeto doble eco}
	 \item \textcolor{red}{Objeto}
	\end{enumerate}
  \end{column}
\end{columns}
\end{frame}
%%==================================================================F FIN
\begin{frame}
  \titlepage
\end{frame}

  %================================================================== Logo On/Off
\LogoOff

%==================================================================FRONT PAGE AND TOC
% For article only
\mode<presentation:0>{\thispagestyle{empty}\maketitle}

% For presentation only
\mode<presentation| article:0| handout:0>{
    \begin{frame}<article:0>[label=portada]
    \titlepage
    \end{frame}%Fin del frame
}

% For handout only
\mode<handout>{
  \begin{frame}[label=portada]
    \maketitle
  \end{frame}
}

%% TABLE OF CONTENTS
% \begin{frame}[label=toc]
%     \mode<article:0>{\frametitle{Contents}}
%     \mode<presentation>{\small}
%     \tableofcontents[hidesubsections]
% \end{frame}

%%==================================================================Sc
\section{Clasificación mediante splines}
%%==================================================================Sb
\begin{frame}
  \frametitle{Objetivos y metodología}
  \begin{enumerate}
    \item Objetivo
    \begin{itemize}
       \item Generar un DTM, un modelo de \alert{vegetación}, un modelo de
         \alert{edificios}...
    \end{itemize}
    \item ¿Cómo?
    \begin{itemize}
      \item Interpolación de \alert{superficies} mediante un ajuste de mínimos
            cuadrados de splines utilizando una norma de Tychonov
          \item Estudio de pendientes: \alert{Morfología}
          \item \alert{Segmentación} de objetos
    \end{itemize}
  \end{enumerate}
\end{frame}
%%==================================================================Sb
\subsection{Splines}
%%==================================================================F
\begin{frame}
    \frametitle{Spline bilineares y bicúbicas}
    \begin{center}
        \includegraphics[width=0.45\textwidth]{images/bilinear_spline.jpg}
        \includegraphics[width=0.45\textwidth]{images/bicubic_spline.jpg}
        %\includegraphics[width=0.45\textwidth]{images/border_errors.jpg}
    \end{center}
\end{frame}
%%==================================================================Sb
\subsection{Puntos groseros}
%%==================================================================F
\begin{frame}
   \frametitle{Puntos groseros}
   \begin{enumerate}
    \item Hay siempre puntos anómalos debido a diferentes eventos:
     \begin{itemize}
     	\item<1-> \alert{Pájaros}
     	\item<2-> \alert{Nubes de agua}
     	\item<3-> \alert{``Multipath''}...
     \end{itemize}
    \item<4>{\alert{\textexclamdown\textexclamdown Deben ser eliminados!!}} 
   \end{enumerate}
   \begin{picture}(0,100)
       \uncover<1,4>{\put(10,10){\includegraphics[height=30mm]{images/bandada}}}
       \uncover<2>{\put(50,10){\includegraphics[height=30mm]{images/vapor_chimenea}}}
       \uncover<2,4>{\put(150,10){\includegraphics[height=30mm]{images/fumo}}}
       \uncover<3->{\put(280,10){\includegraphics[height=30mm]{images/multipath}}}
    \end{picture}
\end{frame}
%%==================================================================F
\begin{frame}
   \frametitle{Detección de puntos groseros}
   \begin{itemize}
    \item La eliminación se hace mediante una interpolacion con splines 
        bicúbicas bastante lisa y baja resolución.
    \item Aquellos puntos que se alejen más de un umbral determinado son 
        considerados como puntos groseros y eliminados.
    \item El umbral por defecto son \alert{$50~m$}
   \end{itemize}

    \begin{picture}(0,100)
       \only{\put(80,0){\includegraphics[height=35mm]{images/obs_splines}}}
       \only{\put(80,0){\includegraphics[height=35mm]{images/3splines}}}
       \only{\put(80,0){\includegraphics[height=35mm]{images/5splines}}}
       \only{\put(80,0){\includegraphics[height=35mm]{images/7splines}}}
       \only{\put(80,0){\includegraphics[height=35mm]{images/9splines}}}
       \only{\put(80,0){\includegraphics[height=35mm]{images/11splines}}}
    \end{picture}
\end{frame}
%%==================================================================Sb
\subsection{Filtro}
%%==================================================================Sbb
\subsubsection{Detección de bordes}
%%==================================================================F
\begin{frame}
  \frametitle{Detección de bordes}
  \begin{beamerboxesrounded}[shadow=true]{Definición: \emph{Borde}}
       Es una fuerte variación en altura correspondiente a una pequeña variación en planimetría
  \end{beamerboxesrounded}

  \begin{picture}(0,80)
    %\centering
    \begin{tikzpicture}[scale=0.7]
    \draw[thick] (-5,0) -- (2.5,0);
    \filldraw[black!20!white] (-2,0) -- (-2,2) -- (0,3) -- (2,2) -- (2,0) -- cycle;
    \draw (-2,0) -- (-2,2) -- (0,3) -- (2,2) -- (2,0) -- cycle;

    \coordinate[mark coordinate] (S) at (-2,0);
    \coordinate[mark coordinate] (B) at (-2,2);
    \node[anchor=south east] at (S){};
    \node[anchor=north west] at (B){};
    \node[anchor=west] at (-1,1){Objeto};
    \end{tikzpicture}
  \end{picture}
  \begin{picture}(0,80)
     \uncover{\put(175,-5){\includegraphics[height=25mm]{images/height_edge}}}
  \end{picture}
\begin{enumerate}
 \item El análisis de bordes no es fácil porque los puntos no están distribuidos regularmente
 \item Se pueden regularizar los puntos $\Rightarrow$ se rasteriza: 
    \begin{itemize}
        \item \alert{mínimo} y \alert{máximo}
	\item \alert{Media}
	\item\alert{Interpolación}
	\item \alert{\emph{kriging}}...
    \end{itemize}
  \item \alert{Pérdida de información} altimétrica
\end{enumerate}
\end{frame}
%%==================================================================F
\begin{frame}
  \frametitle{Cálculo de bordes}
\begin{enumerate}
  \item Se trabaja con una interpolación con spline bilineares que se ajuste 
    mucho a las observaciones (nuestros puntos)
  \item{Para esta superficie se calcula:}
\begin{itemize}
 \item El \alert{Gradiente} para cada punto: $G_m = \sqrt{G_x^2 + G_y^2} = \sqrt{\left( \dfrac{dz}{dx}\right)^2 + \left( \dfrac{dz}{dy}\right)^2}$
 \item La \alert{dirección normal}: $\vartheta_P = \arctan\left(\dfrac{G_y}{G_x} \right)$
\end{itemize}
\end{enumerate}

  \uncover{\alert{Estos dos parámetros no son suficientes!!}}
\end{frame}
%%==================================================================F
\begin{frame}
  \frametitle{Valor gradiente}
  \begin{enumerate}
   \item El gradiente tiene un valor muy alto en el borde de los objetos pero también el punto exterior más próximo
   \item Debido al ruido en las observaciones, el valor más alto del gradiente \alert{puede ser el equivocado}
  \end{enumerate}
    \begin{picture}(0,110)
        \only{\put(80,0){\includegraphics[height=40mm]{images/gradient_edge}}}
        \only{\put(80,0){\includegraphics[height=40mm]{images/gradient_noise_edge}}}
    \end{picture}
\end{frame}
%%==================================================================F
\begin{frame}
  \frametitle{Cálculo de residuos}%
  \begin{enumerate}
   \item Se trabaja con una interpolación con spline bicúbicas con resolución muy baja 
       para obtener una \alert{superficie muy lisa}%
   \item Se calcula el residuo entre la superficie y las observaciones%
   \item El punto con residuo positivo será nuestro punto \alert{borde} %
  \end{enumerate}%
\uncover{
  \begin{figure}[!b]
    \centering
    \begin{tikzpicture}[scale=0.85]
    %\includegraphics[height=50mm]{images/figure_6}
    \draw[thick] (-5,0) -- (3,0);
    \filldraw[black!20!white] (-2,0) -- (-2,2) -- (0,3) -- (2,2) -- (2,0) -- cycle;
    \draw (-2,0) -- (-2,2) -- (0,3)node[above]{Objeto} -- (2,2) -- (2,0) -- cycle;
    \draw[thick] (-4.5,0.3)node[above]{Interpolaci\'on} -- 
        (-3.5,0.3) .. controls (-1.5,.4) and (-1.5,2) .. (0,2);

    \coordinate[mark coordinate] (N) at (-2,0);
    \coordinate[mark coordinate] (P) at (-2,2);
    \node[anchor=south east] at (N){--};
    \node[anchor=north west] at (P){+};

    \node[anchor=west] at (-0.5,1){+ Borde};
    \node[anchor=west] at (-0.5,0.5){-- \  Terreno};
    \end{tikzpicture}
  \end{figure}
}
\end{frame}
%%==================================================================F
\begin{frame}
  \frametitle{Identificación final de bordes}%
  Para que un punto sea considerado como el borde de un objeto se debe cumplir:
  \begin{enumerate}
	\item El gradiente debe ser mayor que un cierto umbral dado por el usario
	\item Si el gradiente es grande pero \alert{sin} llegar al umbral, además se debe verificar que:
	\begin{itemize}
	 \item El residuo es \alert{positivo}
	 \item La dirección normal no se desvía mas de un valor dado.
	\end{itemize}
  \end{enumerate}
\end{frame}
%%==================================================================Sbb
\subsubsection{Region Growing}
%%==================================================================F
\begin{frame}
  \frametitle{``Region Growing''}
\begin{beamerboxesrounded}[shadow=true]{Hipótesis}
El interior del objeto será siempre igual o más alto que los bordes (su altura media)
\end{beamerboxesrounded}

\begin{enumerate}
 \item El objetivo en este paso es reconocer el \alert{interior} de los objetos
 \item Para cada borde se construye un \alert{conj. convexo} y en su interior se ejecuta un algoritmo de \alert{\emph{region growing}} para verificar la hipótesis
\end{enumerate}
\end{frame}
%%==================================================================F
% \begin{frame}
%  \frametitle{``Region Growing''}
% \begin{figure}[t]
% \begin{center}
%  \begin{tikzpicture}[scale=0.2, node distance=1.25cm, auto]
%   \tiny
%    % Region growing
%    \node[io] (semilla) {Elegir Semilla};
%    \node[block, below of=semilla] (este) {Moverse a celda Este};
%    \node[decision, below of=este] (visitadaE) {¿Visitada?};
%    \node[block, below of=visitadaE, node distance=1cm] (norte) 
%        {Moverse a celda Norte};
%    \node[decision, below of=norte] (visitadaN) {¿Visitada?};
%    \node[block, below of=visitadaN, node distance=1cm] (visitada) 
%        {Marcar como visitada};
%    \node[block, right of=visitada, node distance=2cm] (retroceder) {Retroceder};
%    \node[decision, right of=visitadaE, node distance=1.5cm] (valor1E) 
%        {¿Valor = 1?};
%    \node[decision, left of=visitadaN, node distance=1.5cm] (valor1N) 
%        {¿Valor = 1?};
%    \node[decision, above of=retroceder] (EsSemilla) {¿Es la semilla?};
%    \node[io, right of=EsSemilla, node distance=1.5cm] (convhull) 
%        {Crear soporte compacto};

%    % Flujo
%    \path[line] (semilla) -- node{}(este);
%    \path[line] (este) -- node{}(visitadaE);
%    \path[line] (visitadaE) -- node{si}(norte);
%    \path[line] (visitadaE) -- node{no}(valor1E);
%    \path[line] (valor1E) |- node[near start]{si}(este);
%    \path[line] (valor1E) |- node[near start]{no}(norte);
%    \path[line] (norte) -- node{}(visitadaN);
%    \path[line] (visitadaN) -- node{si}(visitada);
%    \path[line] (visitadaN) -- node{no}(valor1N);
%    \path[line] (valor1N) |- node[near start]{si}(este);
%    \path[line] (valor1N) |- node[near start]{no}(visitada);
%    \path[line] (visitada) -- node{}(retroceder);
%    \path[line] (retroceder) -- node{}(EsSemilla);
%    \path[line] (EsSemilla) -- node[near start]{si}(convhull);
%    \path[line] (EsSemilla) |- node[near start]{no}(este);
%    \path[line] (convhull) |- node{}(semilla);
%  \end{tikzpicture}
% \end{center}
% \end{figure}
% \end{frame}
%%==================================================================F
\begin{frame}
  \frametitle{Primera clasificación}
Los puntos viene clasificados por su posición con respecto al terreno:
\begin{enumerate}
 \item \alert{Objeto}: Si están dentro del conj. convexo y su altura es mayor que la altura media del borde
 \item \alert{Terreno}: En cualquier otro caso.
\end{enumerate}

Y por el tipo de impulso:
\begin{enumerate}
 \item \alert{Único impulso}: Si solo se ha recibido un eco de ese punto
 \item \alert{Doble impulso}: Si se han recibido más impulsos
\end{enumerate}
\end{frame}
%%==================================================================Sbb
\subsubsection{Corrección}
%%==================================================================F
\mode<beamer>{
  \pgfdeclareimage[width=0.4\textwidth]{fumo}{images/fumo}
  \pgfdeclareimage[width=0.4\textwidth]{fumo}{images/google_techos}
}
\begin{frame}
  \frametitle{Errores en la clasificación}
  \begin{columns}
    \begin{column}{0.5\textwidth}
      \begin{enumerate}
    	\item<1-> La hipótesis de las alturas falla
      \item<2-> Confusión de alturas
    	\item<3-> Se han identificado bordes pertenecientes al terreno pero no a objetos
     \end{enumerate}
    \end{column}
   \begin{column}{0.45\textwidth}
   \begin{figure}[h!]
   \centering
      \only<1>{\includegraphics[height=35mm]{images/fumo}}
      \only<2>{\includegraphics[height=35mm]{images/google_techos}}
      \only<3>{\includegraphics[height=35mm]{images/terrazas}}
   \end{figure}
   \end{column}
  \end{columns}
\end{frame}
%%==================================================================F
\begin{frame}
  \frametitle{Corrección final}
Se utilizan los puntos \alert{Terreno único impulso} para realizar una
interpolación con splines bilineares y resolucion baja para obtener una
superficie bastante lisa.
  \begin{minipage}{0.5\textwidth}
    \begin{enumerate}
      \item Si un punto terreno está lo suficientemente alejado de la superficie se reclasifica como \alert{objeto}
      \item Si un punto objeto está lo suficientemente cercano a la superficie se reclasifica como \alert{terreno}
    \end{enumerate}
  \end{minipage}%
  ~%
  \begin{minipage}{0.45\textwidth}
   \begin{figure}[h!]
   \centering
      \includegraphics[height=20mm]{images/correct_objeto}\\
      \includegraphics[height=20mm]{images/correct_terreno}
   \end{figure}
  \end{minipage}
\end{frame}
%%==================================================================Sb
\subsection{Vegetación}
%%==================================================================F
\begin{frame}
  \frametitle{Motivaciones}
  \begin{enumerate}
    \item El filtro anterior clasifica entre \alert{terreno} Vs. \alert{objeto}: 
  \begin{itemize}
    \item Participó en el test del ISPRS: Buenos resultados
    \item Incompleta: \alert{NO distingue vegetación}
  \end{itemize}
    \item Detectar vegetación en una nube ALS es importante para
      \alert{estudios hidrológicos},
      \alert{modelización urbana}, \alert{estudios
      forestales}, \alert{etc\ldots}
    \item Desarrollo del filtro de vegetación porque:
    \begin{itemize}
    \item Existe ya una estructura adecuada de datos
    \item Dado que está bajo licencia libre, su publicación también deberá ser
      libre
    \item Completar el algoritmo que participó en el ISPRS
    \end{itemize}
  \end{enumerate}
\end{frame}
%%==================================================================F
\begin{frame}
  \frametitle{Clasificación de vegetación}
  \begin{enumerate}
    \item Los datos de partida son los clasificados como terreno u objecto
    (tanto único como doble impulso).
    \item Se crea una máscara ráster.
    \item Podemos buscar:
      \begin{enumerate}
        \item \alert{Edificios}
          \begin{itemize}
            \item valor celda = 1 si al menos un punto ``objeto único impulso''
              cae en ella.
            \item valor de celda = 0 en otro caso
          \end{itemize}
        \item \alert{Vegetación}
          \begin{itemize}
            \item valor celda = 1 si al menos un punto ``terreno doble impulso''
              cae en ella.
            \item valor de celda = 0 en otro caso
          \end{itemize}
      \end{enumerate}
  \end{enumerate}
\end{frame}
%%==================================================================F
\begin{frame}
  \frametitle{Clasificación de vegetación}
  \begin{enumerate}
    \item Se aplica un algoritmo de ``region growing'' a la mascara raster
        para crear conj. convexos. 
      \item Las \alert{dimensiones} y la relación \alert{área/perímetro}
        se usan para la clasificación de los segmentos:
    \begin{itemize}
        \item Objetos pequeños o muy estrechos no son edificios
        \item Puntos ``terreno doble impulso'' son considerados como vegetación
        \item Puntos ``objeto doble impulso'' fuera de los segmentos son
          considerados como vegetación
        \item El resto siempre es considerado edificio
    \end{itemize}
  \end{enumerate}
\end{frame}
%%==================================================================F
\begin{frame}
  \frametitle{Árbol de clasificación del filtro vegetación}
\begin{figure}[t!]
\begin{center}
\scriptsize
\begin{tikzpicture}[scale=0.75,grow=right, sloped,
    level 1/.style={sibling distance = 3cm, level distance = 3cm}, 
    level 2/.style={sibling distance = 1.5cm, level distance = 3cm},
    level 3/.style={sibling distance = 0.75cm, level distance = 2.5cm},
    punto/.style={text width=7em, text centered},
    mascara/.style={text centered, text width=5em},
    end/.style={circle, minimum width=3pt,fill, inner sep=0pt}]
  \node {Punto}
    child{ node[punto] {Terreno único impulso} 
       child{ node [end, label=right:{\textbf{Terreno}}] {}
              edge from parent node[above] {\scriptsize{Siempre}}
       }
    }
    child{ node[punto] {Terreno doble impulso}
       child{ node[mascara] {Máscara Terreno} 
          child{ node [end, label=right:{\textbf{Terreno}}] {}
                 edge from parent node[below] {\scriptsize{Fuera}}
          }
          child{ node [end, label=right:{\textbf{Vegetación}}] {}
                 edge from parent node[above] {\scriptsize{Dentro}}
          }
       }
       child{ node[mascara] {Máscara Objeto}
          child{ node [end, label=right:{\textbf{Vegetación}}] {}
                 edge from parent node[above] {\scriptsize{Siempre}}
          }
       }
    }
    child{ node[punto] {Objeto doble y único impulso}
       child{ node[mascara] {Máscara Terreno} 
          child{ node [end, label=right:{\textbf{Objeto}}] {}
                 edge from parent node[below] {\scriptsize{Fuera}}
          }
          child{ node [end, label=right:{\textbf{Vegetación}}] {}
                 edge from parent node[above] {\scriptsize{Dentro}}
          }
       }
       child{ node[mascara] {Máscara Objeto}
          child{ node [end, label=right:{\textbf{Vegetación}}] {}
                 edge from parent node[below] {\scriptsize{Fuera}}
          }
          child{ node [end, label=right:{\textbf{Objeto}}] {}
                 edge from parent node[above] {\scriptsize{Dentro}}
          }
       }
    }
;
\end{tikzpicture}
\caption{Árbol de decisión para la clasificación de los puntos en terreno,
vegetación y edificio.}
\label{fig:decision_tree}
\end{center}
\end{figure}
\end{frame}
%%==================================================================F
\begin{frame}
  \frametitle{Clasificación de la vegetación}
  \begin{minipage}{0.45\textwidth}
    \begin{figure}[h!]
      \centering
      \includegraphics[width=0.95\textwidth]{images/olbia}
    \end{figure}
  \end{minipage}
  \begin{minipage}{0.45\textwidth}
    \begin{figure}[h!]
      \centering
      \includegraphics[width=0.95\textwidth]{images/olbia_veg_build}
    \end{figure}
  \end{minipage}
\end{frame}
%%==================================================================F
\begin{frame}
  \frametitle{Resultados}
  \begin{enumerate}
    \item El filtro vegetación está fuertemente afectado por el filtro terreno:
    \begin{itemize}
      \item Altos errores del tipo I (error de omisión) debido
        a bordes de edificios mal clasificados como vegetación.
      \item Errores de tipo II (error de comisión) debido
        a zonas de densa vegetación tomada como edificios o incluso zonas
        terreno clasificadas como objetos.
    \end{itemize}
    \item Alrededor del 90\% de los puntos se clasifican correctamente.
    \begin{itemize}
      \item Amplias zonas vegetales son correctamente clasificadas, así como la
        mayoría de árboles individuales. 
      \item La figura y forma de los eficios está siempre bien definida.
    \end{itemize}
  \end{enumerate}
\end{frame}
%%==================================================================F
\begin{frame}
  \frametitle{Resultados}
  \begin{itemize}
    \item Errores de \alert{tipo I} (descartar puntos edificio):
      \alert{$16.6\%$}
    \item Errores de \alert{tipo II} (Aceptar puntos vegetación como
      edificio): \alert{$7.5\%$}
      \item \alert{Error Total}: \alert{$10.6\%$}
    \end{itemize} 
     \begin{table}[b!]
    \centering
    \begin{tabular}{lccccr}
      \multicolumn{6}{c}{\bfseries{Matriz de Confusión}} \\
      \hline
      \textbf{Clasif.} & \multicolumn{2}{c}{Edificios} & 
        \multicolumn{2}{c}{Vegetación} & Total \\
      \cline{2-6}
      \textbf{Verdad} & \multicolumn{2}{c}{Puntos (\%)} & 
        \multicolumn{2}{c}{Puntos (\%)} & Total \\
      \hline
      Edif. & \multicolumn{2}{c}{179678 (83.4)} & 
        \multicolumn{2}{c}{34919 (\alert{16.6})} & 214597 \\
      Veg. & \multicolumn{2}{c}{37590 (\alert{7.5})} & 
        \multicolumn{2}{c}{388709 (92.5)} & 426299 \\
      Total & \multicolumn{2}{c}{217268 (--)} & 
      \multicolumn{2}{c}{423628 (--)} & 640896\footnote{Las estadísticas están referidas a aquellos puntos con una pre-clasificación distinta de terreno único impulso} \\
      \hline
    \end{tabular}
    \end{table}
\end{frame}
%%==================================================================F
\begin{frame}
  \frametitle{Solapes}
  \begin{enumerate}
    \item Durante las interpolaciones, se divide toda la región en teselas
      solapadas
      \begin{itemize}
        \item Permite la computación de la interpolación de modo más rápido
        \item Elimina los problemas de alocación de memoria
      \end{itemize}
    \item En las zonas de solape
    \begin{itemize}
        \item Asegura la continuidad de toda la interpolación
        \item Introduce una discontinuidad en la primera derivada: Errores en el
          mapa de aspectos en zonas con ausencia de datos
      \end{itemize}
  \end{enumerate}
\end{frame}
%%==================================================================F
\begin{frame}
  \frametitle{Solapes}
    \begin{figure}[h!]
\centering
\begin{tikzpicture}[scale=0.50]
  \filldraw[fill=gray!40!white]   
    (4,0) -- (4,5) -- (9,5) -- (9,0) -- cycle;
  \draw (0,0) -- (0,5) -- (5,5) -- (5,0) -- cycle;
  \draw
    (0,4) node[anchor=north west] {Tesela 3} -- 
    (0,9) node[anchor=north west] {Tesela 1} -- 
    (5,9) node[anchor=north west] {Tesela 2} -- 
    (5,4) node[anchor=north west] {Tesela 4} -- cycle;
  \draw (4,4) -- (4,9) -- (9,9) -- (9,4) -- cycle;
  \draw
    (0,5) -- (4,5) -- node[below=.1cm] {1} 
    (5,5) -- node[below=0.1cm] {2} 
    (8,5) -- node[below=0.1cm] {3} (9,5); 
  \draw
    (0,1) -- (4,1) -- node[below=.1cm] {7} 
    (5,1) -- node[below=.1cm] {8} 
    (8,1) -- node[below=.1cm] {9} (9,1); 
  \draw
    (0,4) -- (4,4) -- node[below=.75cm] {4} 
    (5,4) -- node[below=.75cm] {5} 
    (8,4) -- node[below=.75cm] {6} (9,4); 
  \draw (4,0) -- (4,9); 
  \draw (8,0) -- (8,9); 
    % Continuacion de teselas en lineas de puntos
  \draw[thick] (4,0) -- (4,5) -- (9,5) -- (9,0) -- cycle;
  \draw[dotted,thick] (9,0) -- (10,0);
  \draw[dotted,thick] (9,1) -- (10,1);
  \draw[dotted,thick] (9,4) -- (10,4);
  \draw[dotted,thick] (9,5) -- (10,5);
  \draw[dotted,thick] (9,9) -- (10,9);
  \draw[dotted,thick] (0,-1) -- (0,0);
  \draw[dotted,thick] (4,-1) -- (4,0);
  \draw[dotted,thick] (5,-1) -- (5,0);
  \draw[dotted,thick] (8,-1) -- (8,0);
  \draw[dotted,thick] (9,-1) -- (9,0);
    \end{tikzpicture}
    \end{figure}
\end{frame}
%%==================================================================F
\begin{frame}
  \frametitle{Solapes}
    \begin{figure}[h!]
      \centering
      \only<1>{\includegraphics[width=0.6\textwidth]{images/errores_aspect}}
      \only<2>{\includegraphics[width=0.6\textwidth]{images/errores_aspect_mascara}}
      \only<3>{\includegraphics[width=0.6\textwidth]{images/error_interpolacion_solapes}}
    \end{figure}
\end{frame}
%%==================================================================F
%\begin{frame}
%  \frametitle{Computación}
%    \begin{table}[h!]
%      \centering
%    \begin{tabular}{ccrrr}
%    \toprule
%    \multicolumn{5}{c}{\bfseries Interpolación Bilinear} \\
%    \hline
%    \bfseries Paso de Spline & \bfseries Número de teselas & \bfseries
%    tradicional & \bfseries teselado & \bfseries gain \\ 
%    % (m) & (NSxEW) & (time) & (time) & \\
%    \hline\hline
%    1 & 16x16 & 2h16m42s & 1h33m47s & 31,4\% \\ 
%    2 & 8x8 & 32m33s & 22m18s & 31,5\% \\ 
%    4 & 4x4 & 8m03s & 6m10s & 23,4\%\\ 
%    8 & 2x2 & 2m18s & 2m30s & -8,7\% \\
%    16 & 1x1 & 1m01s & 1m41s & -65,6\% \\
%    \bottomrule
%    \end{tabular}
%    \end{table}
%\end{frame}
%%==================================================================FIN
\begin{frame}
 \titlepage
\end{frame}

\end{document}
