%% PAQUETES
\usepackage[utf8]{inputenx}
\usepackage[spanish]{babel}
\usepackage{amsmath}
\usepackage{amsthm}
\usepackage{amssymb}
\usepackage{amsfonts}   
\usepackage{pifont}     % Para tickYes y tickNo
\usepackage{graphicx}
\usepackage{epsfig}
\usepackage{pdflscape}
\usepackage{enumerate}
\usepackage{color}
\usepackage{beamerfoils}
\usepackage{beamerprosper}
\usepackage{xspace}
\usepackage{booktabs}
\usepackage{multicol}
\usepackage{multirow}
\usepackage{tikz}
\usepackage{pgflibraryshapes}
\usepackage{wasysym}
\usepackage{float}
\usepackage{listings}

%% CONFIGURACIÓN DE TIKZ
% Librerías cargadas
\usetikzlibrary{arrows}
%\usetikzlibrary{calc,fadings,decorations.pathreplacing,arrows,intersections,
%    decorations.markings,external,shapes}

% Configuración de Tikz y Pgfplot
\tikzset{%
    >=stealth, % option for nice arrows
    inner sep=0pt,%
    outer sep=3pt,%
    mark coordinate/.style={inner sep=0pt,outer
    sep=0pt,minimum size=3pt,
    fill=black,circle}%
    }

% Librerías de tiKz
    %\usetikzlibrary{}

    % Define block styles
\tikzstyle{io} = [trapezium,trapezium left angle=70,trapezium right angle=-70,
  minimum height=0.9cm, draw, fill=blue!20, text width=4.5em, 
  text badly centered, node distance=2cm, inner sep=0pt]
\tikzstyle{decision} = [diamond, draw, fill=blue!20, text width=5.5em, 
  text badly centered, node distance=2cm, inner sep=0pt]
\tikzstyle{block} = [rectangle, draw, fill=blue!20, text width=5em, 
  text centered, rounded corners, minimum height=4em]
\tikzstyle{line} = [draw, -latex']
\tikzstyle{cloud} = [draw, ellipse,fill=red!20, node distance=2cm, 
    minimum height=2em]


%% TEMAS BASE
\usetheme{Boadilla}
\setbeamertemplate{headline}{
\begin{beamercolorbox}{section in head/foot}
\vskip2pt\insertnavigation{\paperwidth}\vskip2pt
\end{beamercolorbox}
\includegraphics[width=\paperwidth]{images/ucm_logo}}

\usecolortheme{lily}

\useinnertheme{circles}

\definecolor{ZurichBlue}{rgb}{.255,.41,.884}
\setbeamercolor{alerted text}{fg=ZurichBlue}
\definecolor{Comentario}{rgb}{0.0,0.5,0.0}

%% CONFIGURACIÓN GEENERAL DE HIPERVÍNCULOS
\hypersetup{%
    baseurl={roberto.antolin@forestry.gsi.gov.uk},
    pdftitle={Filtro de datos LiDAR con GRASS},%
    pdfauthor={Roberto Antol\'{i}n S\'anchez},%
    pdfcreator={Roberto Antol\'{i}n S\'anchez},%
    pdfproducer=PDFLaTeX,%
    pdfsubject={LiDAR GRASS data filtering},%
    pdfkeywords={filtros, splines, LiDAR, GRASS}%
  }

\mode<presentation>{\hypersetup{pdfpagemode=FullScreen}}

%% COMANDOS
% Tipos de datos
\newcommand{\file}[1]{\path{#1}\xspace} %Referencias a ficheros
\newcommand{\extr}[1]{\emph{#1}\xspace} %Palabras extranjeras
\mode<article>{\newcommand{\soft}[1]{\textsf{#1}\xspace} }%Referencias a software
\mode<presentation>{\newcommand{\soft}[1]{\textrm{#1}\xspace} }%Referencias a software
\newcommand{\sigl}[1]{\textsc{#1}\xspace} %Siglas
\newcommand{\nota}[1]{\textcolor{nota}{�#1?}\xspace}
\newcommand{\frametitlee}[1]{\mode<presentation| handout| article:0>{\frametitle{#1}}}
\newcommand{\figuree}[2]{
  \mode<article>{
    \begin{figure}[H]
    \centering
    }
    #1
    \mode<article>{
    \caption{#2}
    \end{figure}
  }
}

% Atajos
\newcommand{\inet}{\extr{Internet}}
\newcommand{\html}{\sigl{html}}
\newcommand{\web}{\extr{web}}
\newcommand{\software}{\extr{software}}
\newcommand{\hardware}{\extr{hardware}}
\newcommand{\tickNo}{\color{red}\ding{55}}
\newcommand{\tickYes}{\color{green!50!black}\checkmark}

%% PREFERENCIAS PARA LISTINGS (código)
\lstdefinestyle{mio}{
  basicstyle=\ttfamily\scriptsize,%
  keywordstyle=\bfseries\color{ZurichBlue},%
  commentstyle=shape\color{green},%
  stringstyle=\color{red!50!black},
}

\lstdefinestyle{latex}
{
language={[latex]tex},
keywordstyle=\color{resalta!50!black},
commentstyle=\color{fondoO!50!green}\itshape
}

\lstdefinestyle{shell}{
  language=sh,
  basicstyle=\ttfamily\scriptsize,%
  morekeywords={lasinfo,lasdiff,lasmerge,las2txt,txt2las,las2ogr,laszip,
    lasview, lasboundary, lasgrid, lasduplicate, lasoverlap,
    las2dem, las2las, lastile, lasground, lasheight, lasclassify},
  keywordstyle=\bfseries\color{ZurichBlue},
  commentstyle=\color{green!35!black}%
}

\lstdefinestyle{grass}{                                                                     
  language=sh,
  basicstyle=\ttfamily\scriptsize,%
  morekeywords={v.lidar.edgedetection,v.lidar.growing, v.lidar.correction,v.surf.bspline,v.outlier},
  keywordstyle=\bfseries\color{ZurichBlue},classoffset=1,
  commentstyle=shape\color{green!75!black},%
  morekeywords={GRASS6 (stuttgart):, >}
  keywordstyle=\color{green!50!black}, classoffset=0
}

%
\lstdefinestyle{matlab}{
  basicstyle=\scriptsize\ttfamily,
  breaklines=true,
  commentstyle=\color{Comentario},
  keywordstyle=\bfseries\color{blue},
  language=Matlab,
  %xleftmargin=\parindent,
  identifierstyle=\bfseries
}

\lstset{style=mio,
   tabsize=4,%
   inputencoding=utf8,%latin1,%
   extendedchars=true,%
   breaklines=true,%
   showstringspaces=false,
   backgroundcolor=\color{gray!20!white}
   }

%% DATOS DEL DOCUMENTO
\title[Tutorial: procesar datos LiDAR con GRASS]
    {Tutorial: procesar datos LiDAR con GRASS}

\author[R. Antolín]
    {Roberto Antolín Sánchez}

\institute[FR]{
  Investigador y desarrollador \\
	Forest Research \\
  Northern Research Station \\
	Roslin EH25 9SY \\
	\texttt{roberto.antolin@forestry.gsi.gov.uk}\\
}

%% FECHA DE LA PRESENTACIÓN
\date[15-03-2014]{15 de Marzo de 2014}

\AtBeginSection[]
{
  \begin{frame}[shrink=20]
    \frametitle{Table of Contents}
    \tableofcontents[currentsection,hidesubsections]
  \end{frame}
}

\begin{document}
  %%==================================================================FRONT PAGE AND TOC
%% For article only
%\mode<presentation:0>{\thispagestyle{empty}\maketitle}
%
%% For presentation only
%\mode<presentation| article:0| handout:0>{
%    \begin{frame}<article:0>[label=portada]
%    \titlepage
%    \end{frame}%Fin del frame
%}
%
%% For handout only
%\mode<handout>{
%  \begin{frame}[label=portada]
%    \maketitle
%  \end{frame}
%}
%
%%% TABLE OF CONTENTS
%\begin{frame}[label=toc]
%    \mode<article:0>{\frametitle{Contents}}
%    \mode<presentation>{\small}
%    \tableofcontents%[hidesubsections]
%\end{frame}
%
%%==================================================================F PORTADA
\begin{frame}[label=portada]
  \titlepage
\end{frame}%Fin del frame
%%==================================================================S LOS DATOS
\section{Los datos}
%%==================================================================F 
\begin{frame}
\frametitle{La nube de puntos}
Metadatos
\begin{itemize}
 \item Sensor LiDAR: \alert{Optech ATLM}
 \item Datos disponibles: Primer y último impulso.
 \item Localización: Centro de Stuttgart (estación f.c.)
 \item Resolución: $0.91$ puntos$/m^2$ 
 \item Resolución media: $0.87$ puntos$/m^2$
 \item Numero de puntos: 
  \begin{enumerate}
    \item $259030$ de primeros retornos
    \item $259030$ de últimos retornos
   \end{enumerate}
\end{itemize}
\end{frame}
%%==================================================================F 
\pgfdeclareimage[width=0.5\textwidth]{stuttgart}{images/stuttgart_google}
\begin{frame}
\frametitle{Estación f.c. de Stuttgart}
 \begin{columns}
  \begin{column}{0.45\textwidth}
	Coordenadas UTM-32 (MBR):
	\begin{itemize}
	 \item Norte = $5403764~m$
	 \item Sur = $5403188~m$
	 \item Oeste = $513632~m$
	 \item Este = $513118~m$
	\end{itemize}
   \end{column}
  \begin{column}{0.55\textwidth}
    \begin{center}
	\begin{tikzpicture}
    	  \pgftext[bottom,left,at={\pgfpointxy{0}{0}}]{\pgfuseimage{stuttgart}}
	\end{tikzpicture}
	\tiny{\textcopyright 2008 Google -- imágenes}
    \end{center}
   \end{column}
 \end{columns}
\end{frame}
%%==================================================================F 
\pgfdeclareimage[width=0.7\textwidth]{las}{images/las_format}
\begin{frame}
\frametitle{Estandard binario \emph{LAS} del ASPRS:}
\begin{itemize}
 \item Encabezado público
 \item Registros de longitud variable
 \item Puntos: X, Y, Z, Intensidad, clasificación, dirección de la pasada del vuelo,...
    \begin{center}
	\begin{tikzpicture}
    	  \pgftext[bottom,left,at={\pgfpointxy{0}{0}}]{\pgfuseimage{las}}
	\end{tikzpicture}
   \end{center}
 \item Para un conversor \alert{abierto} del estándar \emph{LAS}:
\beamergotobutton{\url{http://liblas.org/}}
\end{itemize}
\end{frame}
%%==================================================================F 
\frame[containsverbatim]{%
\frametitle {Posible formato ASCII}
\begin{center}
\begin{verbatim}
              X          Y       Z
          513628.82|5403190.04|291.40
          513628.85|5403191.47|291.24
          513628.95|5403192.93|294.31
          513628.99|5403194.45|294.17
          513628.97|5403196.05|291.26
          513629.01|5403197.58|291.24
          513629.05|5403199.10|291.25
          513629.09|5403200.53|291.26
                        ...
\end{verbatim}
\end{center}
}
%%==================================================================S
\section{Preprocesado de datos}
%%==================================================================F
\begin{frame}
 \frametitle{Geographic Resources Analysis Support System\hfill 
   \includegraphics[width=0.7cm]{images/grasslogo_transp_big.png}} 
 \begin{itemize}
    \item Conocido como \alert<1>{GRASS}: GIS para manipulación y 
      análisis de datos geoespaciales.
    \item Como es un programa de \alert{códico abierto} con un 
      montón de librerias, es muy adecuado para la implementación 
      de nuestras propias herramientas.
    \item Escrito en C/C++, lo que permite implementar programas 
      computacionalmente pesados $\Rightarrow$ \alert{ideal} 
      para trabajar con LiDAR.
    \item GRASS $\geq$ 6 incluye indexación topológica $\Rightarrow$ 
      \alert{fatal} para trabajar con LiDAR.
    \item Hay versiones para GNU/Linux, MS-Windows, Mac-OSX, SUN, etc. 
      \alert{\textexclamdown\textexclamdown Todos lo pueden utilizar!!}
    \item Más información en: \beamergotobutton{\url{http://grass.osgeo.ogr}} 
  \end{itemize}
\end{frame}
%%==================================================================Sb
\subsection{Importando nube de puntos}
%%==================================================================F 
\defverbatim[colored]\vlidar{
\begin{lstlisting}[style=shell]
# Mostrar la informacion del LAS
v.in.lidar -p input=CSite4_orig.las
# Crear una nueva ``localizacion'' con la informacion del LAS
v.in.lidar -i input=CSite4_orig.las location=Stuttgart
# Cerrar y abrir GRASS en la nueva localizacion
# Importar el archivo LAS sin topologia ni tabla asociada
v.in.lidar -tb input=CSite4_orig.las output=stuttgartLAS
\end{lstlisting}
}
\begin{frame}
 \frametitle{Importar como .LAS: \LARGE\path{v.in.las}}
 \vlidar
\end{frame}
%%==================================================================F 
\begin{frame}
  \frametitle{Importar como LAS: \LARGE\path{v.in.las}}
 \begin{center}
 \includegraphics[height=50mm]{images/vinascii_strips}
 \end{center}
\end{frame}
%%==================================================================F 
\defverbatim[colored]\head{
\begin{lstlisting}[style=shell]
  $ # Copiar python en ``/Documents and Settings/aula1/Desktop/taller LiDAR/''
  $ # Descargar el conversor de datos de http://tinyurl.com/ca3vdfb
  $ # Formato ASCII de los datos
  $ python isprs2lf.py CSite4_orig.txt
  $ head CSite4_orig.txt
\end{lstlisting}
}
\begin{frame}[fragile,shrink=2]
\frametitle {Posible formato ASCII}
\head
\begin{center}
\begin{verbatim}
              X          Y       Z
          513628.82|5403190.04|291.40
          513628.85|5403191.47|291.24
          513628.95|5403192.93|294.31
          513628.99|5403194.45|294.17
          513628.97|5403196.05|291.26
\end{verbatim}
%          513629.01|5403197.58|291.24
%          513629.05|5403199.10|291.25
%          513629.09|5403200.53|291.26
%                        ...
\end{center}
\end{frame}
%%==================================================================F 
\defverbatim[colored]\extension{
\begin{lstlisting}[style=shell]
GRASS6: > # Primero conocemos la extension del archivo
GRASS6: > r.in.xyz -sg in=CSite4_orig_all.txt
GRASS6: > # Fijo la region con la que trabajar
GRASS6: > g.region n=5403764 s=5403188 e=513632 w=513118
GRASS6: > # Genero una visualizacion de las pasadas
GRASS6: > # considerando para cada celda el numero de puntos
GRASS6: > r.in.xyz in=CSite4_orig.txt out=pasadas method=n fs='|'
\end{lstlisting}
}
\begin{frame}
 \frametitle{Importar como raster: \LARGE\path{r.in.xyz}}
 \extension
\end{frame}
%%==================================================================F 
\defverbatim[colored]\xyz{
\begin{lstlisting}[style=shell]
# Generar un raster con la informacion LiDAR
r.in.xyz in=CSite4_orig_first.txt out=stuttgart_first fs='|' x=1 y=2 z=3
r.in.xyz in=CSite4_orig_last.txt out=stuttgart_last fs='|' x=1 y=2 z=3
\end{lstlisting}
}
\begin{frame}
 \frametitle{\LARGE\path{r.in.xyz}}
 \xyz
 \pause
 \begin{center}
 \includegraphics[height=50mm]{images/rinxyz_first}~
 \includegraphics[height=50mm]{images/rinxyz_last}
 \end{center}
\end{frame}
%%==================================================================F 
\defverbatim[colored]\vascii{
\begin{lstlisting}[style=shell]
# Generar dos vectoriales con la informacion LiDAR
# de ultimo y primer impulso
v.in.ascii -tbzr in=CSite4_orig_first.txt out=stuttgart_first fs='|' x=1 y=2 z=3
v.in.ascii -tbzr in=CSite4_orig_last.txt out=stuttgart_last fs='|' x=1 y=2 z=3
\end{lstlisting}
}
\begin{frame}
 \frametitle{Importar como vectorial: \LARGE\path{v.in.ascii}}
 \vascii
 \pause
 \begin{center}
 \includegraphics[height=40mm]{images/vinascii_strips}
 \end{center}
\end{frame}
%%==================================================================Sb
\subsection{Eliminando puntos groseros}
%%==================================================================F 
\defverbatim[colored]\voutlier{
\begin{lstlisting}[style=shell]
# Limpiamos el primer impulso
v.outlier in=stuttgart_first out=station_first \
> outlier=station_first_out soe=10 son=10 thres_o=25
# Limpiamos el ultimo impulso
v.outlier in=stuttgart_last out=station_last \
> outlier=station_last_out soe=10 son=10 thres_o=30
\end{lstlisting}
}
\begin{frame}[shrink=05]
 \frametitle{\path{v.outlier}}
\begin{beamerboxesrounded}[shadow=true]{\textbf{\path{v.lidar.edgedetection}}\texttt{ input=name output=name outlier=name [soe=value]
[son=value] [lambda\_i=value] [thres\_o=value]}}
\begin{itemize}
 \item \textbf{input}: name of the input vector map
 \item \textbf{output}: name of the output vector map
 \item \textbf{soe}: Interpolation spline step value in e-w direction
 \item \textbf{son}: Interpolation spline step value in n-s direction
 \item \textbf{lambda\_i}: Thychonov regularization weigth. (default: 0.1)
 \item \textbf{thres\_o}: Threshold for the outliers. (default: 50)
\end{itemize}
\end{beamerboxesrounded}
\voutlier
\end{frame}
%%==================================================================F 
\begin{frame}
 \frametitle{Visualización de \LARGE\path{v.outlier}}
 \begin{center}
 \includegraphics[height=50mm]{images/voutlier_first}~
 \includegraphics[height=50mm]{images/routlier_first}
 \end{center}
\end{frame}
%%==================================================================F 
\begin{frame}
 \frametitle{Visualización de \LARGE\path{v.outlier}}
 \begin{center}
 \includegraphics[height=50mm]{images/voutlier_last}~
 \includegraphics[height=50mm]{images/routlier_last}
 \end{center}
\end{frame}
%%==================================================================S
\section{Clasificación de la nube de puntos}
%%==================================================================Sb
\subsection{Detección de borde de objetos}
%%==================================================================F 
\defverbatim[colored]\vedge{
\begin{lstlisting}[style=shell]
# Deteccion de bordes
v.lidar.edgedetection in=station_last see=4 sen=4 lambda_g=0.1 \
> out=station_edge
\end{lstlisting}
}
\begin{frame}[shrink=02]
 \frametitle{\path{v.lidar.edgedetection}}
\begin{beamerboxesrounded}[shadow=true]{\textbf{\path{v.lidar.edgedetection}}\texttt{ input=name output=name first=name see=value sen=value [lambda\_g=value] [tgh=value] [tgl=value] [theta\_g=value] [lambda\_r=value]}}
\begin{itemize}
 \item \textbf{input}: name of the input vector map
 \item \textbf{output}: name of the output vector map
 \item \textbf{see}: Interpolation spline step value in e-w direction
 \item \textbf{sen}: Interpolation spline step value in n-s direction
 \item \textbf{lambda\_g}: Regularization weight in gradient evaluation (default: 0.01)
 \item \textbf{tgh}: High gradient threshold for edge classification (default: 6)
 \item \textbf{tgl}: Low gradient threshold for edge classification (default: 3)
 \item \textbf{theta\_g}: Angle range for same direction detection (default: 0.26)
 \item \textbf{lambda\_r}: Regularization weight in residual evaluation (default: 2)
\end{itemize}
\end{beamerboxesrounded}
\vedge
\end{frame}
%%==================================================================F 
\pgfdeclareimage[width=0.5\textwidth]{edge}{images/edge}
\defverbatim[colored]\dedge{
\begin{lstlisting}[style=shell]
$ d.vect station_edge size=1 cat=1 color=yellow
$ d.vect station_edge size=1 cat=2 color=red
\end{lstlisting}
}
\begin{frame}
 \frametitle{Visualización de \LARGE\path{v.lidar.edgedetection}}
    \dedge
\begin{columns}
  \begin{column}{0.65\textwidth}
    \begin{center}
	\begin{tikzpicture}
    	  \pgftext[bottom,left,at={\pgfpointxy{0}{0}}]{\pgfuseimage{edge}}
	\end{tikzpicture}
   \end{center}
  \end{column}
  \begin{column}{0.35\textwidth}
    \begin{enumerate}
      \item \textcolor{red}{Borde}
      \item \textcolor{yellow!95!black}{Terreno}
    \end{enumerate}
  \end{column}
\end{columns}
\end{frame}
%%==================================================================Sb
\subsection{Relleno de los límites de los objetos}
%%==================================================================F 
\defverbatim[colored]\vgrow{
\begin{lstlisting}[style=shell]
# IMPORTANTE: Disminuir primero la resolucion
g.region -p res=2
v.lidar.growing in=station_edge out=station_grow \
> first=station_first
\end{lstlisting}
}
\begin{frame}[shrink=03]
 \frametitle{\path{v.lidar.growing}}
\begin{beamerboxesrounded}[shadow=true]{\textbf{\path{v.lidar.growing}}\texttt{ input=name output=name first=name [tj=value] [td=value]}}
\begin{itemize}
 \item \textbf{input}: name of the input vector map
 \item \textbf{output}: name of the output vector map
 \item \textbf{first}: name of the input vector map of the first pulse points
 \item \textbf{tj}: Threshold in planimetric units for considering two measurements as corresponding first and double pulse. (default
0.2)
 \item td: Threshold in height units for classifying a point as double pulse. (default: 0.6)
\end{itemize}
\end{beamerboxesrounded}
\vgrow
\end{frame}
%%==================================================================F 
\pgfdeclareimage[width=0.45\textwidth]{grow}{images/grow}
\defverbatim[colored]\dgrow{
\begin{lstlisting}[style=shell]
$ d.vect station_grow layer=2 size=1 cat=1 color=yellow
$ d.vect station_grow layer=2 size=1 cat=2 color=green
$ d.vect station_grow layer=2 size=1 cat=3 color=blue
$ d.vect station_grow layer=2 size=1 cat=4 color=red
\end{lstlisting}
}
\begin{frame}
 \frametitle{Visualización de \LARGE\path{v.lidar.growing}}
 \dgrow
\begin{columns}
  \begin{column}{0.65\textwidth}
    \begin{center}
	\begin{tikzpicture}
    	  \pgftext[bottom,left,at={\pgfpointxy{0}{0}}]{\pgfuseimage{grow}}
	\end{tikzpicture}
   \end{center}
  \end{column}
  \begin{column}{0.35\textwidth}
	\begin{enumerate}
	 \item \textcolor{yellow!95!black}{Terreno}
	 \item \textcolor{green}{Terreno doble eco}
	 \item \textcolor{blue!90!black}{Objeto doble eco}
	 \item \textcolor{red}{Objeto}
	\end{enumerate}
  \end{column}
\end{columns}
\end{frame}
%%==================================================================Sb
\subsection{Corrección final}
%%==================================================================F 
\defverbatim[colored]\vcorr{
\begin{lstlisting}[style=shell]
v.lidar.correction in=station_grow sce=60 scn=60 lambda_c=2 \
> tcl=0.1 out=station_corr terrain=station_terr
\end{lstlisting}
}
\begin{frame}[shrink=04]
 \frametitle{\path{v.lidar.correction}}
\begin{beamerboxesrounded}[shadow=true]{\textbf{\path{v.lidar.correction}}\texttt{ input=name  output=name  terrain=name  sce=value  scn=value  [lambda\_c=value]  [tch=value]  [tcl=value]}}
\begin{itemize}
 \item \textbf{input}: name of the input vector map
 \item \textbf{output}: name of the output vector map
 \item \textbf{terrain}: name of the terrain output vector map
 \item \textbf{sce}: Interpolation spline step value in e-w direction
 \item \textbf{scn}: Interpolation spline step value in n-s direction
 \item \textbf{lambda\_c}: Regularization weight in reclassification evaluation. (default 1)
 \item \textbf{tch}: High difference threshold (terrain/object). (default 2)
 \item \textbf{tcl}: Low differnce threshold (object/terrain). (default 1)
\end{itemize}
\end{beamerboxesrounded}
\vcorr
\end{frame}
%%==================================================================F 
\pgfdeclareimage[width=0.45\textwidth]{corr}{images/correction}
\defverbatim[colored]\dcorr{
\begin{lstlisting}[style=shell]
$ d.vect station_corr layer=2 size=1 cat=1 color=yellow
$ d.vect station_corr layer=2 size=1 cat=2 color=green
$ d.vect station_corr layer=2 size=1 cat=3 color=blue
$ d.vect station_corr layer=2 size=1 cat=4 color=red
\end{lstlisting}
}
\begin{frame}
 \frametitle{Visualización de \LARGE\path{v.lidar.correction}}
 \dcorr
\begin{columns}
  \begin{column}{0.65\textwidth}
    \begin{center}
	\begin{tikzpicture}
    	  \pgftext[bottom,left,at={\pgfpointxy{0}{0}}]{\pgfuseimage{corr}}
	\end{tikzpicture}
   \end{center}
  \end{column}
  \begin{column}{0.35\textwidth}
	\begin{enumerate}
	 \item \textcolor{yellow!95!black}{Terreno}
	 \item \textcolor{green}{Terreno doble eco}
	 \item \textcolor{blue!90!black}{Objeto doble eco}
	 \item \textcolor{red}{Objeto}
	\end{enumerate}
  \end{column}
\end{columns}
\end{frame}
%%==================================================================F FIN
\begin{frame}
  \titlepage
\end{frame}

\end{document}
